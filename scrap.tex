
The $G^o_b$-invariance (Left invariance?) of $\mathcal{L}$ implies that the Lagrangian $\mathcal{L} = \mathcal{L}(q_m, \dot{q}_m,^bV^o_b)$ does not explicitly depend on the base configuration $g^o_b$, i.e., the base shape space variables are cyclic (why? not necessarily? does that not relate to ...'s work from 90s?). %Noting that the Lagrange-Poincare equations are local equations (meaning?), since the coordinates correspond to a (local) trivialization of the reduced shape space bundle(?).
Since these equations are independent of $g^o_b$, they locally drop to the quotient(see Appendix \ref{Appendix:geometric}).
Let $Q$ denote the $n+6$-dimensional configuration manifold of the space manipulator. An element $q \in Q$ consists of a set of velocities of the base spacecraft and joint parameters of the manipulator.
It is possible to write
down a useful explicit form of the reduced Lagrangian for mechanical systems
(see, e.g., Murray [1995] and Ostrowski [1995, 1998]). We shall also
see a related, even simpler version of this in the next section in the context
of Routh reduction.
Suppose we have a Lagrangian on L on TQ that is invariant under the
action of G on Q. We define the reduced Lagrangian l : TQ/G → R to be
l(r, ξ, r˙) = L(r, g
−1g, ˙ r, g
−1 ˙ g) , (3.11.5)
where (ξ = g−1g˙, r, r˙) are local coordinates on TQ/G. Here the ξ are referred
to as body velocities or velocities with respect to the body frame,
while ξs = ˙gg−1 = Adgξ are referred to as spatial velocities. See Marsden
and Ratiu [1999] for further details.
For a mechanical system the Lagrangian has the form
For a G-invariant mechanical Lagrangian the reduced
Lagrangian may be written in the form
l(r, r˙, ξ) =
1
2
(ξT , r˙T )

I IA
ATI m(r)

ξ˙
r

− V (r) ,
where here I is the local form of the locked inertia tensor and A is the local
form of the mechanical connection.

Setting Ω = ξ + Ar˙ we obtain the block diagonal form
l = lΩ(r, r˙,Ω) =
1
2
(ΩT , r˙T )

I 0
0 m − ATIA

Ω˙
r

−V (r) . (3.11.13)
We can then compute the equations of motion in these reduced coordinates.
Define the generalized momentum
p = ∂l
∂ξ
= Iξ + IAr˙ .
Suppose also that there is a G-invariant forcing F = (Fα, Fa), the components
corresponding to base and fiber directions, respectively. Then the
3.11 The Lagrange–Poincar´e Equations 149
reduced equations of motion are
g
−1g˙ = ξ = −Ar˙ + I
−1p , (3.11.15)
p˙ = ad∗
ξp + F , (3.11.16)
˜M
¨r + ˙rT ˜ C(r) ˙ r + ˜N + ∂V
∂r
= T(r)F,
where ˜M (r) = m(r)−AT (r)I(r)A(r), and ˜ C(r) represents reduced Coriolis
and centrifugal terms,
r˙TC˜(r)r˙ = Cαβγr˙αr˙β =
1
2
#
∂ ˜Mαβ
∂rγ + ∂ ˜Mαγ
∂rβ
− ∂ ˜Mγβ
∂rα

r˙βr˙α , (3.11.18)
while
)
N˜, δr
*
=
&
p, dA( ˙ r, δr) − [A( ˙ r,A(δr)] +

I
−1p,A(δr)

+
1
2
∂I−1p
∂r
(δr)
'
(3.11.19)
and (T(r)F)α = Fα −FaAa
α, Fa = Fbgba
, with gba
denoting the lifted action
of the group.
Noting that the Lagrange-Poincare equations are local equations (meaning?), since the coordinates correspond to a (local) trivialization of the reduced shape space bundle(?).





Using the exponential map to parameterize the relative configuration manifolds of these 1-parameter subgroups of $\SE$, leads to the standard product of exponentials
formula\cite{Chhabra2014a}. The product of exponentials formula, first introduced by Brockett\cite{brockett1984robotic} and further developed by Murray et al.\cite{murray1994mathematical}, with roots in Lie group and screw theory, provides an extension of the described parameterizations to calculate the kinematics of multi-body systems.